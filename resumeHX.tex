% resumeHX.tex
%
% (c) 2002 Huafeng Xu
%
\documentclass[11pt]{article}
\usepackage{fullpage}
%\usepackage{mathptmx}
%\usepackage[T1]{fontenc}
%\usepackage{times}
\usepackage{hyperref}
\textheight=9.0in
\voffset=0.5in
\pagestyle{empty}
\raggedbottom
\raggedright
\setlength{\tabcolsep}{0in}

\begin{document}

\begin{tabular*}{6.5in}{l@{\extracolsep{\fill}}r}
\textbf{Huafeng Xu}, Ph. D.  & \\
Cryptope Biosciences & \\
9715 71st Avenue & \\
Forest Hills, NY 11375 & 1-917-968-8176 \\ 
U. S. A. & huafeng@gmail.com \\
\end{tabular*}

\vspace{0.05in}
\begin{tabular*}{6.5in}{l@{\extracolsep{\fill}}r}
\hline \\
\end{tabular*}

\vspace{0.02in}
\textit{Seeking a collaborative environment dedicated to advancing science and technology that impact human health and society}

\vspace{0.1in}
{\large \textbf{Education}}

\begin{itemize}

	\item
	\begin{tabular*}{6in}{l@{\extracolsep{\fill}}r}
		B.S. / Chemistry, 1997 & \\
		\textbf{Beijing University} & Beijing, P. R. China \\
	\end{tabular*}

	\item
	\begin{tabular*}{6in}{l@{\extracolsep{\fill}}r}
		M.S. / Chemistry, 1998 & \\
		\textbf{Columbia University} & New York, New York \\
	\end{tabular*}

	\item 
	\begin{tabular*}{6in}{l@{\extracolsep{\fill}}r}
		Ph.D. / Chemical Physics, 2001 & \\
		\textbf{Columbia University}   & New York, New York \\
	\end{tabular*}

\end{itemize}

{\large \textbf{Professional Experience}}

\begin{itemize}

\item
  \begin{tabular*}{6in}{l@{\extracolsep{\fill}}r}
     Founder & January 2018 -- present \\
     \textbf{Cryptope Biosciences} & New York, New York \\
  \end{tabular*}

  \begin{itemize}
    \item
      I have been developing a computational discovery platform
      that integrates latest machine learning techniques into
      physics-based modeling, in order to enable accurate design of molecules
      with desired structural and functional properties.
    \item
      I have secured a \$75K joint grant with Professor John Chodera of
      Memorial Sloan Kettering Cancer Center to develop a new
      computational method for predicting neoantigen epitopes.
    \item
      I am in preliminary discussion with biotech and pharmaceutical companies 
      for potential collaborations to deploy Cryptope's discovery platform
      and to co-develop Cryptope's cancer vaccine program.
  \end{itemize}

\item 
  \begin{tabular*}{6in}{l@{\extracolsep{\fill}}r}
    Senior Scientist &  June 2005 -- December 2017 \\
    \textbf{D. E. Shaw Research} & New York, New York \\
  \end{tabular*}
  
  \begin{itemize}
    \item
      I led a team of chemists and computer scientists in developing
      methods and software for computing protein-ligand binding free
      energies, which have become pivotal in the internal drug
      discovery projects.  Our implementation of the free energy
      methods laid the foundation for the successful commercial FEP+
      software by Schrodinger, which is widely deployed in major
      pharmaceutical companies.
    \item
      I led the structural immunology and antigen design effort,
      managing the internal computational work, external
      collaborations with Harvard Medical School, Duke Vaccine Center,
      and Novartis Vaccines, and the outsourced experimental
      studies by six contract research organizations around the globe
      (USA, Germany, and China).
    \item 
      Together with a team of computer scientists, I developed the
      original DESMOND molecular dynamics (MD) simulation program, currently
      the fastest parallel MD program in the world.
    \item
      I was the longest-serving hiring manager for both junior scientific
      associates and senior computational chemists, in which role I
      drafted job advertisements, devised recruitment strategies,
      reviewed and interviewed candidates, made hiring decisions, and
      was instrumental to growing the company's scientific team in the
      past decade.
  \end{itemize}

\item
  \begin{tabular*}{6in}{l@{\extracolsep{\fill}}r}
    Visiting Postdoctoral Scholar (Advisor: Kenneth A. Dill) & September, 2003 -- June, 2005 \\
    \textbf{University of California, San Francisco} & San Francisco, California \\
  \end{tabular*}
  
  Research projects include:
  
  \begin{itemize}
  \item
    a new theory of hydrophobic effect and solvation thermodynamics;
  \item
    the role of solvation thermodynamics in protein folding and protein-protein interactions;
  \item
    path ensemble formulation of non-equilibrium statistical mechanics.
  \end{itemize}
  
\item
  \begin{tabular*}{6in}{l@{\extracolsep{\fill}}r}
    Research Scientist & September 2001 -- September, 2003 \\
    \textbf{3-Dimensional Pharmaceuticals, Johnson \& Johnson} & Cranbury, New Jersey \\
  \end{tabular*}
  
  Research projects include:
  
  \begin{itemize}
  \item
    machine-learning algorithms for dimensionality reduction of nonlinear
    data sets;
  \item
    methodology development for conformational sampling of small molecules,
    protein-ligand docking, and virtual screening for discovery of drug
    leads;
  \item
    web-based computational tools in support of the company's internal drug
    pipeline.
  \end{itemize}
  
\item
  \begin{tabular*}{6in}{l@{\extracolsep{\fill}}r}
    Summer Intern & June -- September 2000 \\
    \textbf{T. J. Watson Research Center, IBM} & Yorktown Heights, New York \\
  \end{tabular*}
  
  Developed the core molecular dynamics simulation program for
  the Blue Gene project.
  
\item
  \begin{tabular*}{6in}{l@{\extracolsep{\fill}}r}
    Ph.D. Student (Advisor: Bruce J. Berne) & September 1997 -- September 2001 \\
    \textbf{Columbia University} & New York, New York \\
  \end{tabular*}
  
  Research projects include:
  
  \begin{itemize}
  \item
    stochastic sampling of molecular systems;
  \item
    fast computation of long-range electrostatic forces and
    integration methods in molecular dynamics simulations with
    multiple time scales;
  \item
    hydrogen bond structure and dynamics in the solvation shell of
    biomolecules.
  \end{itemize}
  
  Thesis: \textit{New Methods to Accelerate Biomolecular Simulations
    and Their Applications.}
  
\end{itemize}

{\large \textbf{Teaching Experience}}
\begin{itemize}

\item
  \begin{tabular*}{6in}{l@{\extracolsep{\fill}}r}
    Instructor, Chemical Kinetics Laboratory & Spring, 2004 \\
    \textbf{University of California, San Francisco} & San Francisco, California \\
  \end{tabular*}
  
\item 
  \begin{tabular*}{6in}{l@{\extracolsep{\fill}}r}
    Teaching Assistant & September 1997 -- July 1999 \\
    \textbf{Columbia University} & New York, New York \\
  \end{tabular*}
  
  T.A. in \textit{Computer-Aided Teaching in General Chemistry}
  (Undergraduate), 1997; Head T.A. in \textit{Intensive General
    Chemistry Laboratory} (Undergraduate), 1998; T.A. in
  \textit{Statistical Mechanics} (Graduate), 1998;
  \textit{Quantum Mechanics} (Graduate), 1999.

\end{itemize}

{\large \textbf{Professional Associations}}
\begin{itemize}
\item American Chemical Society, American Physics Society, Biophysical Society
\end{itemize}

{\large \textbf{Honors}}
\nopagebreak
\begin{itemize}
\item
Supercomputing Conference Best Paper Award, 2006.
\item
Miller Award for Excellence in Teaching, Department of Chemistry, Columbia University, 1998.
\end{itemize}

{\large \textbf{Bibliography}}

\begin{enumerate}
\item
\textbf{Huafeng Xu} Cochaperones enable Hsp70 to fold proteins like a Maxwell's demon. \textit{Submitted}.
\item
\textbf{Huafeng Xu}, Timothy Palpant, Cody Weinberger, and David E. Shaw, Characterizing receptor flexibility to predict mutations that lead to human adaptation of influenza hemagglutinin. \textit{Manuscript in preparation}.
\item
\textbf{Huafeng Xu}, Timothy Palpant, Qi Wang, and David E. Shaw, Design of antigens to present a tumor-specific cryptic epitope. \textit{Manuscript in preparation}.
\item
Albert C. Pan, \textbf{Huafeng Xu}, Timothy Palpant, and David E. Shaw, Quantitative characterization of the binding and unbinding of millimolar drug fragments with molecular dynamics simulations. \textit{Journal of Chemical Theory and Computation}, \textbf{13}(7), 3372-3377, 2017.
\item
\textbf{Huafeng Xu} and David E. Shaw, A simple model of multivalent adhesion and its application to influenza infection. \textit{Biophysical Journal}, \textbf{110}(1), 218-233, 2016.
\item
\textbf{Huafeng Xu}, Aaron G. Schmidt, Timothy O'Donnell, Matthew D Therkelsen, Thomas B Kepler, M Anthony Moody, Barton F Haynes, Hua‐Xin Liao, Stephen C Harrison, and David E Shaw, Key mutations stabilize antigen‐binding conformation during affinity maturation of a broadly neutralizing influenza antibody lineage. \textit{Proteins: Structure, Function, and Bioinformatics}, \textbf{83}(4), 771-780, 2015.
\item
Aaron G Schmidt, \textbf{Huafeng Xu}, Amir R Khan, Timothy O’Donnell, Surender Khurana, Lisa R King, Jody Manischewitz, Hana Golding, Pirada Suphaphiphat, Andrea Carfi, Ethan C Settembre, Philip R Dormitzer, Thomas B Kepler, Ruijun Zhang, M Anthony Moody, Barton F Haynes, Hua-Xin Liao, David E Shaw, and Stephen C Harrison, Preconfiguration of the antigen-binding site during affinity maturation of a broadly neutralizing influenza virus antibody. \textit{Proceedings of the National Academy of Sciences}, \textbf{110}(1), 264-269, 2013.
\item
Ron O Dror, Robert M Dirks, JP Grossman, \textbf{Huafeng Xu}, David E Shaw, Biomolecular simulation: a computational microscope for molecular biology. \textit{Annual Review of Biophysics}, \textbf{41}, 429-452, 2012.
\item
Cristian Predescu, Ross A Lippert, Michael P Eastwood, Douglas Ierardi, \textbf{Huafeng Xu}, Morten Ø Jensen, Kevin J Bowers, Justin Gullingsrud, Charles A Rendleman, Ron O Dror, David E Shaw, Computationally efficient molecular dynamics integrators with improved sampling accuracy. \textit{Molecular Physics}, \textbf{110}(9-10), 967-983, 2012.
\item
Robert M Dirks, \textbf{Huafeng Xu}, David E Shaw, Improving sampling by exchanging hamiltonians with efficiently configured nonequilibrium simulations. \textit{Journal of Chemical Theory and Computation}, \textbf{8}(1), 162-171, 2011.
\item
Ron O Dror, Daniel H Arlow, Paul Maragakis, Thomas J Mildorf, Albert C Pan, \textbf{Huafeng Xu}, David W Borhani, David E Shaw, Activation mechanism of the β2-adrenergic receptor. \textit{Proceedings of the National Academy of Sciences, U.S.A.}, \textbf{108}(46), 18684-18689, 2011.
\item
Ron O. Dror, Albert C. Pan, Daniel H. Arlow, David W. Borhani, Paul Maragakis, Yibing Shan, \textbf{Huafeng Xu}, and David E. Shaw, Pathway and Mechanism of Drug Binding to G Protein–Coupled Receptors. \textit{Proceedings of the National Academy of Sciences, U.S.A.}, \textbf{108}(32), 13118-13123, 2011.
\item
Dimitris K. Agrafiotis, \textbf{Huafeng Xu}, Fangqiang Zhu, Deepak Bandyopadhyay, Pu Liu, Stochastic Proximity Embedding: Methods and Applications. \textit{Molecular Informatics}, \textbf{29}(11), 758-770, 2010.
\item
Yibing Shan, Markus A. Seeliger, Michael P. Eastwood, Filipp Frank, \textbf{Huafeng Xu}, Morten \O. Jensen, Ron O. Dror, John Kuriyan, and David E. Shaw, A Conserved Protonation-Dependent Switch Controls Drug Binding in the Abl Kinase. \textit{Proceedings of the National Academy of Sciences, U.S.A.}, \textbf{106}(1), 139–144, 2009.
\item
Daniel K. Shenfeld, \textbf{Huafeng Xu}, Michael P. Eastwood, Ron O. Dror, and David E. Shaw, Minimizing Thermodynamic Length to Select Intermediate States for Free-Energy Calculations and Replica-Exchange Simulations. \textit{Physical Review E}, \textbf{80}(4), 046705(1-4), 2009.
\item
Morten \O. Jensen, Ron O. Dror, \textbf{Huafeng Xu}, David W. Borhani, Isaiah T. Arkin, Michael P. Eastwood, and David E. Shaw, Dynamic Control of Slow Water Transport by Aquaporin 0: Implications for Hydration and Junction Stability in the Eye Lens. \textit{Proceedings of the National Academy of Sciences, U.S.A.}, \textbf{105}(38), 14430–14435, 2008.
\item
Paul Maragakis, Kresten Lindorff-Larsen, Michael P. Eastwood, Ron O. Dror, John L. Klepeis, Isaiah T. Arkin, Morten \O. Jensen, \textbf{Huafeng Xu}, Nikola Trbovic, Richard A. Friesner, Arthur G. Palmer III, and David E. Shaw, Microsecond Molecular Dynamics Simulation Shows Effect of Slow Loop Dynamics on Backbone Amide Order Parameters of Proteins. \textit{Journal of Physical Chemistry B}, \textbf{112}(19), 6155–6158, 2008.
\item
Isaiah T. Arkin, \textbf{Huafeng Xu}, Morten \O. Jensen, Eyal Arbely, Estelle R. Bennett, Kevin J. Bowers, Edmond Chow, Ron O. Dror, Michael P. Eastwood, Ravenna Flitman-Tene, Brent A. Gregersen, John L. Klepeis, István Kolossváry, Yibing Shan, and David E. Shaw, Mechanism of Na$^+$/H$^+$ Antiporting. \textit{Science}, \textbf{317}(5839), 799–803, 2007.
\item
Kevin J. Bowers, Edmond Chow, \textbf{Huafeng Xu}, Ron O. Dror, Michael P. Eastwood, Brent A. Gregersen, John L. Klepeis, István Kolossváry, Mark A. Moraes, Federico D. Sacerdoti, John K. Salmon, Yibing Shan, and David E. Shaw, Scalable Algorithms for Molecular Dynamics Simulations on Commodity Clusters. \textit{Proceedings of the ACM/IEEE Conference on Supercomputing (SC06)}, New York, NY: IEEE, 2006.
\item
\textbf{Huafeng Xu} and Ken A. Dill, Water's Hydrogen Bonds in the Hydrophobic Effect: a Simple Model. \textit{Journal of Physical Chemistry B.}, \textbf{109}(49), 23611-23617, 2005.
\item
\textbf{Huafeng Xu} and Dimitris K. Agrafiotis, Nearest neighbor search in general metric spaces using a tree data structure with a simple heuristic. \textit{Journal of Chemical Information and Computer Sciences}, \textbf{43}(6), 1933-1941, 2003.
\item
Michael A. Farnum, \textbf{Huafeng Xu} and Dimitris K. Agrafiotis, Exploring the nonlinear geometry of protein homology. \textit{Protein Science}, \textbf{12}(8), 1604-1612, 2003.
\item
\textbf{Huafeng Xu}, Sergei Izrailev and Dimitris K. Agrafiotis, Conformational sampling by self-organization. \textit{Journal of Chemical Information and Computer Sciences}, \textbf{43}(4), 1186-1191, 2003.
\item
Dimitris K. Agrafiotis and \textbf{Huafeng Xu}, A geodesic framework for analyzing molecular similarities. \textit{Journal of Chemical Information and Computer Sciences}, \textbf{43}(2), 475-484, 2003.
\item
Dimitris K. Agrafiotis and \textbf{Huafeng Xu}, A self-organizing principle for learning nonlinear manifolds. \textit{Proceedings of the National Academy of Sciences, U.S.A.}, \textbf{99}(25), 15869-15872, 2002.
\item
\textbf{Huafeng Xu} and Dimitris K. Agrafiotis, Retrospect and prospect of virtual screening in drug discovery. \textit{Current Topics in Medicinal Chemistry}, \textbf{2}(12), 1305-1320, 2002.
\item
\textbf{Huafeng Xu}, Harry Stern and B.~J. Berne, Can water polarizability be ignored in hydrogen bond kinetics? \textit{Journal of Physical Chemistry B.}, \textbf{106}(8), 2054-2060, 2002.
\item
\textbf{Huafeng Xu} and B.~J. Berne, Hydrogen bond kinetics in the solvation shell of a polypeptide. \textit{Journal of Physical Chemistry B.}, \textbf{105}(48), 11929-11932, 2001.
\item
Ruhong Zhou, Edward Harder, \textbf{Huafeng Xu} and B.~J. Berne, Efficient multiple time step method for use with Ewald and Particle-Particle Particle-Mesh Ewald for large biomolecular systems. \textit{Journal of Chemical Physics}, \textbf{115}(5), 2348-2358, 2001.
\item
\textbf{Huafeng Xu} and B.~J. Berne, Multicanonical Jump Walk Annealing: An efficient method for geometric optimization. \textit{Journal of Chemical Physics}, \textbf{112}(6), 2701-2708, 2000.
\item
\textbf{Huafeng Xu} and B.~J. Berne, Multicanonical Jump Walking: A method for efficiently sampling rough energy landscapes. \textit{Journal of Chemical Physics}, \textbf{110}(21), 10299-10306, 1999.

\end{enumerate}

{\large \textbf{Patents}}

\begin{enumerate}
\item
Dimitris Agratiotis, \textbf{Huafeng Xu}, Sergei F. Izrailev, and Francis R. Salemme, \textit{Conformational sampling by self-organization}, United States Patent Application 10/519638.
\item
Dimitris Agrafiotis, \textbf{Huafeng Xu}, and Francis R. Salemme, \textit{Methods, systems, and computer program products for representing object realtionships in a multidimensional space}, United States Patent Application 10/517739.
\item
Ruhong Zhou, Edward Harder, \textbf{Huafeng Xu}, and B.~J. Berne, \textit{System and method for molecular dynamics simulations}, United States Patent No. 7096167.
\end{enumerate}

\end{document}





